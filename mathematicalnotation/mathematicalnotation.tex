\documentclass{beamer}
\usefonttheme{professionalfonts}
\mode<presentation>
{
	\usetheme{Warsaw}
	\setbeamercovered{transparent}
}
\usepackage[english]{babel}
\usepackage[utf8]{inputenc}
\usepackage{gensymb}
\usepackage{amsmath}
\usepackage{amssymb}
\usepackage{cancel}
\usepackage{soul}
\usepackage{xcolor}
\usepackage{mathtools}
\usepackage{hyperref}
\usefonttheme{serif}
\newcommand{\sequiv}{\stackrel{\mathclap{\normalfont\mbox{s}}}{\equiv}}
\title[BRINMANOLA]{A Brief Introduction to the Use of Mathematical Notation as a Language}
\author[VALEFOR, VARIK]{VALEFOR, VARIK \textless varikvalefor@aol.com \!\!\textgreater}
\begin{document}
	\maketitle
	\begin{frame}{Preface: The Curse of Knowledge}
		Curse of knowledge
		\begin{itemize}
			\item Do not remember being a newbie very well
			\item ``What do newbies not know?''
			\item ``What \textit{do} newbies know?''
			\item Confusion should be voiced
		\end{itemize}
\end{frame}
	\begin{frame}{About the Speaker}
		VARIK VALEFOR
			\begin{itemize}
				\item Intelligence analyst
				\begin{itemize}
				\item \textbf{\textit{Not}} a computer programmer!!!
				\item Can only answer a few questions re: work
			\end{itemize}
				\item Diverse set of skills
			\begin{itemize}
				\item Mathematics!
				\begin{itemize}
					\item Can't add numbers or do other low-level things
					\item Dove into calculus in the sixth grade
					\item Concepts understood and applied well
					\item Computers made for a reason
				\end{itemize}
				\item Computer programming
				\begin{itemize}
					\item Obsoletes low-to-mid-level mathematical tasks
				\end{itemize}
				\item Hosting kick-ass parties near Thanksgiving
				\begin{itemize}
					\item Egg splatter
					\item Roomba
					\item Battery explosion
				\end{itemize}
			\end{itemize}
		\end{itemize}
	\end{frame}

	\section{What Is Mathematical Notation?}
		\begin{frame}
			\begin{center}
				What Is Mathematical Notation?
			\end{center}
		\end{frame}
		\subsection{Overview}
			\begin{frame}{What Is Mathematical Notation?}
				Mathematical notation
				\begin{itemize}
					\item Used internally by mathematicians
					\item Different from the ``human languages''
					\begin{itemize}
						\item Simplicity: No adjectives, etc.
						\item Not directly spoken
					\end{itemize}
					\item \textit{Not limited to numbers}
					\begin{itemize}
						\item Neither is mathematics
					\end{itemize}
				\end{itemize}
			\end{frame}
		\subsection{Advantages of Mathematical Notation}
			\begin{frame}{Advantages of Mathematical Notation}
				Extreme precision\\
				\begin{itemize}
					\item Homonyms not a thing here
				\end{itemize}
				Extreme terseness\\
				Variables and functions locally defined
				\begin{itemize}
					\item English formally uses global definitions in dictionaries
				\end{itemize}
				Facilitates analytical thought
				\begin{itemize}
					\item Extremely dry
					\item No loaded words
					\item Arguments easily followed and contested
				\end{itemize}
				Humour still possible
				\begin{itemize}
					\item Surrealism
					\item Weird verbosity
					\item ``No shit, SHERLOCK.''
				\end{itemize}
			\end{frame}
			\begin{frame}{Example of a Joke}
				\[
					\frac{\textrm{d}}{\textrm{d}x} \frac{1}{x} =
          \frac{\cancel{\textrm{d}}}{\cancel{\textrm{d}}x} \frac{1}{x} =
          \frac{}{x} \frac{1}{x} =
          -\frac{1}{x^2}.
				\]
			\end{frame}
		\subsection{Disadvantages of Mathematical Notation}
			\begin{frame}{Disadvantages of Mathematical Notation}
				Not directly spoken
				\begin{itemize}
					\item Converted to ``human language''
				\end{itemize}
				Not very readable for the ``uninitiated''
				\begin{itemize}
					\item Maybe this video can be of use!
				\end{itemize}
				No global definitions
				\begin{itemize}
					\item Definitions \textit{\textbf{must}} be read
				\end{itemize}
				Typos can be \textit{really} bad
			\end{frame}
	\section{Grammar of Mathematical Notation}
		\begin{frame}
			\begin{center}
				The Grammar of Mathematical Notation
			\end{center}
		\end{frame}
			\subsection{Atoms}
				\begin{frame}{Nouns}
					Literals
					\begin{itemize}
						\item $2$, $3.14$, etc.
					\end{itemize}
					Variables
					\begin{itemize}
						\item $\pi$, $e$, $\textit{asshat}$, etc.
						\item Multi-character variable names are valid!
					\end{itemize}
					Functions and operators
					\begin{itemize}
						\item $\sqrt{2}$, $x\left(y\right)$,
            and so on, and so on, and so on, and so on...
					\end{itemize}
					Anything contained within parentheses\\
					Any expression
				\end{frame}
				\begin{frame}{Verbs}
					``Indicators''
					\begin{itemize}
						\item Logical connectives and declarations
						\item ``$=$'', ``$\implies$'', etc.
					\end{itemize}
					Functions
					\begin{itemize}
						\item Yes!
						\item Act as nouns \textit{or} verbs
						\item Example: ``$x\left(y\right).$'' $\sequiv$
              ``$x\left(y\right) \textrm{ IS TRUE}.$''
					\end{itemize}
				\end{frame}
			\subsection{Sentences}
				\begin{frame}{Sentences in Mathematical Notation}
					Sentences
					\begin{itemize}
						\item Contain verb and noun
						\item Must express some sort of idea
						\item ``Can this be literally translated to a proper English sentence?''
					\end{itemize}
				\end{frame}
				\begin{frame}{Examples of Sentences}
					\[
						a := 0! = 1.
					\]
					\[
						\forall z \in \mathbb{N}, \ \left(z + 1\right) \in \mathbb{Z}.
					\]
					\[
						p\left(enis\right).\footnote{$p$ must be a binary function.}
					\]
				\end{frame}
				\begin{frame}{Examples of Non-Sentences}
					\[
						5 + 2
					\]
					\[
						\sqrt{5318008}
					\]
					\[
						90210
					\]
				\end{frame}
		\subsection{Punctuation}
			\begin{frame}{Ending Punctuation}
				Full stops
				\begin{itemize}
					\item Used to end all statements
				\end{itemize}
				Question marks
				\begin{itemize}
					\item \textbf{Not accepted by all men}
					\item Used to indicate uncertainty
					\item Used in place of noun
				\end{itemize}
				$\neg\left(\textrm{Exclamation marks}\right)$
				\begin{itemize}
					\item Used for factorials
					\item Other usage would be a bit confusing
				\end{itemize}
			\end{frame}
			\begin{frame}{Questions}
				Questions in mathematical notation
				\begin{itemize}
					\item Not completely accepted
					\item VARIK wrote this portion of a style guide
					\item Example: ``$X \implies Y?$''
					\item Example: ``$? \ \implies Y.$''
				\end{itemize}
			\end{frame}
	\section{Overview of Common Symbols}
		\begin{frame}
			\begin{center}
				An Overview of Commonly-Used Symbols
			\end{center}
		\end{frame}
		\subsection{Logical Symbols}
			\begin{frame}{Arrows}
				$\implies$ --- implication\\
				\begin{itemize}
					\item ``$A \implies B$.'' $\sequiv$ ``IF $A$, THEN $B$.''
				\end{itemize}
				$\Longleftarrow$  --- ``only if''\\
				\begin{itemize}
					\item ``$A \Longleftarrow B.$'' $\sequiv$ ``IF $B$, THEN $B$.''
				\end{itemize}
				$\iff$ --- ``if and only if'' or ``iff''
				\begin{itemize}
					\item $\left(A \implies B\right) \land \left(B \implies A\right)$
				\end{itemize}
			\end{frame}
			\begin{frame}{Other Logical Symbols}
				``$\exists$'' $\sequiv$ ``there exists''\\
				``$\ni, \ \textrm{s.t.}$'' $\sequiv$ ``such that''\\
				``$\therefore$'' $\sequiv$ ``therefore''\\
				``$\because$'' $\sequiv$ ``because''\\
				``$\land$'' $\sequiv$ ``and''\\
				``$\lor$'' $\sequiv$ ``or''\\
				``$\veebar$'' $\sequiv$ ``xor''\\
				``$\neg$'' $\sequiv$ ``not''
				\begin{itemize}
					\item Example: $\neg\left(1 + 2 = 1 - 2\right)$
				\end{itemize}
				``$\forall$'' $\sequiv$ ``for all''
				\begin{itemize}
					\item Example: $\forall x \in \mathbb{C}, \ x + 1 \neq x.$
				\end{itemize}
			\end{frame}
		\subsection{Sequences and Similar Things}
			\begin{frame}{Sequence/Set Stuff}
				$\in$ --- element of\\
				\begin{itemize}
					\item NOT semantically equivalent to ``$=$''!
					\item Equality statements bidirectional
					\item $\forall a, \ \forall b, \ a = b \iff b = a.$
				\end{itemize}
				$\subset$, $\subseteq$ --- subset\\
				$\#$ --- cardinality
				\begin{itemize}
					\item Example: $\#A = \textrm{ANSWER}
          \left(\textrm{``How many elements does $A$ contain?''}\right)$
				\end{itemize}
				$A_n$ --- element number $n$ of sequence $A$\\
				$\cup$ --- set union\\
				\begin{itemize}
					\item Example: $A \cup B$
				\end{itemize}
				$\cap$ --- set intersection
				\begin{itemize}
					\item Example: $A \cap B$
				\end{itemize}
				$\setminus$ or $-$ --- set difference
				\begin{itemize}
					\item Example: $A \setminus B$ or $A - B$
				\end{itemize}
			\end{frame}
		\subsection{General Stuff}
			\begin{frame}{The Equality Operator}
				Equality operator
				\begin{itemize}
					\item Misunderstood \textit{far} too often
					\item Both sides equal
					\item Means same thing when reversed
					\begin{itemize}
						\item $\left(a + b = c\right) \iff \left(c = b + a\right)$
					\end{itemize}
					\item Improper usage
					\begin{itemize}
						\item $\textrm{SUM}\left(1,2,3,4,5\right)$
						\item $\neg\left(1 + 2 = 3 + 3 = 6 + 4 = 10 + 5 = 15\right).$
					\end{itemize}
				\end{itemize}
			\end{frame}
	\section{Relatively Advanced Stuff}
		\begin{frame}
			\begin{center}
				Relatively Advanced Stuff
			\end{center}
		\end{frame}
		\subsection{Functions}
			\begin{frame}{What Are Functions?}
				Functions
				\begin{itemize}
					\item Map input to output
					\item Can be thought of as ``performing an action''
					\item Extremely versatile
					\item Operate on numbers, Boolean values, sets...
				\end{itemize}
			\end{frame}
			\begin{frame}{Examples}
				\[
					\textrm{colour} : \textrm{OBJ} \mapsto \textrm{SEQ}.
				\]
				\[
					\forall o, \ \textrm{colour}\left(o\right) = \textrm{COLOUR OF OBJECT $o$}.
				\]
			\end{frame}
		\subsection{Piecewise Function Definitions}
			\begin{frame}{Overview}
				Piecewise definitions
				\begin{itemize}
					\item Declare functions
					\item Specifically outline function outputs
					\item Support any sort of input
					\item Equivalent to \texttt{if} statements in CS
				\end{itemize}
			\end{frame}
			\begin{frame}{Example}
				\[
					\forall x \in \mathbb{R}, \ 
					\left|x\right| :=
					\left\{
						\begin{array}{rl}
							x	&	x \geq 0\\
							-x & x < 0
						\end{array}
					\right..
				\]
			\end{frame}
		\subsection{Sets and Sequences}
			\begin{frame}{Sets}
				Sets
				\begin{itemize}
					\item ``Unordered lists''
					\item Denoted by $\left\{\textrm{curly brackets}\right\}$
					\item Can contain zero elements... or be uncountably infinite
					\item Useful whenever dealing with multiple things
					\item Can contain quite literally anything
					\begin{itemize}
						\item Even things which have never existed!
						\item Even things which never \textit{will} exist!!!
					\end{itemize}
					\item Can be clearly defined
					\begin{itemize}
						\item Example: ``$A = \left\{\textrm{MACAULAY CULKIN},\ 
							\textrm{SADDAM HUSSEIN}\right\}$''
					\end{itemize}
					\item Or...
				\end{itemize}
			\end{frame}
			\begin{frame}{Sequences}
				Sequences
				\begin{itemize}
					\item Ordered sets
					\item Denoted by $\left(\textrm{parentheses}\right)$ or
						$\left\langle\textrm{angle brackets}\right\rangle$
					\item Access of individual sets via $A_n$ useful
				\end{itemize}
			\end{frame}
			\begin{frame}{Set-Builder Notation}
				Set-builder notation
				\begin{itemize}
					\item Equal to set which matches conditions
					\item Can return null set
					\item Extremely useful when working with infinite sets
					\item Same thing used for sequences
				\end{itemize}
			\end{frame}
			\begin{frame}{Example of Set-Builder Notation}
				\[
					x^2 = 1 \iff
					x \in \left\{y : y^2 = 1\right\} \iff
					x \in \left\{-1,\ 1\right\}.
				\]
				However...
				\[
					A :=
					\left\{
						\textrm{APPLE} :
						\textrm{COLOUR}\left(\textrm{APPLE}\right) =
						\textrm{GREEN}
					\right\}.
				\]
				Also...
				\[
					P :=
					\left\langle
						a \in \left(\mathbb{N} - \left\{0,1\right\}\right) :
						a \textrm{ IS PRIME}
					\right\rangle.
				\]
			\end{frame}
	\section{Use Cases}
		\subsection{Logical Arguments}
			\begin{frame}{Why Use this Notation in Arguments?}
				Easily dissected
				\begin{itemize}
					\item Facilitates checking for errors
					\item Arguing against becomes easy
				\end{itemize}
			\end{frame}
			\begin{frame}{Example}
				Where $a$ denotes a man, \quad $\forall a, \ 
				\left(k\left(a\right) \iff \textrm{$a$ IS KNOWLEDGABLE}\right) \land
				\left(c\left(a\right) \iff a \textrm{ IS CORRUPT}\right) \land
				\left(p\left(a\right) \iff a \textrm{ IS POWERFUL}\right)$,
				\[
					\forall a, \ k\left(a\right) \implies p\left(a\right).
				\]
				\[
					\forall a, \ p\left(a\right) \implies c\left(a\right).
				\]
				\[
					\therefore \forall a, \ k\left(a\right) \implies c\left(a\right).
				\]
			\end{frame}
		\subsection{General Communication}
			\begin{frame}{Why Use this Notation in General Communication?}
				Mathematical notation...
				\begin{itemize}
					\item Reduces confusion
					\item Saves time
					\item \textit{May} encourage thinking about stuff objectively
					\item Is easily translated to other languages
					\begin{itemize}
						\item Only definitions need be translated
					\end{itemize}
				\end{itemize}
			\end{frame}
			\begin{frame}{Example}
				Refer to the example for the use of mathematical notation in 
        logical arguments.  Both things end up looking fairly similar.
			\end{frame}
	\section{Conclusion}
		\subsection{Conclusion}
			\begin{frame}{Conk'd!}
				NOT COMPREHENSIVE!!!\\
				At least one of these things should apply:
				\begin{itemize}
					\item Spiked reader's \st{drink} interest in mathematics
					\item Encouraged better structuring of arguments
				\end{itemize}
				Recommendations welcomed
			\end{frame}
\end{document}
